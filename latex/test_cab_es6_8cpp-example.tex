\hypertarget{test_cab_es6_8cpp-example}{}\section{test\+\_\+cab\+\_\+es6.\+cpp}
This illustrates how to compute a Gr\"{o}bner basis of the second example in \cite{CaboaraDynAlg}, \[ u + v + y + 32002 ,\\ t + 2 u + z + -3 ,\\ t + 2 v + y + 32002 ,\\ 32002 t + 32002 u + 32002 v + x + 32002 y + 32002 z ,\\ t u x^2 + 26650 y z^3 ,\\ 28222 t y + v z \]


\begin{DoxyCodeInclude}
\textcolor{comment}{/*****************************************************************************\(\backslash\)}
\textcolor{comment}{* This file is part of DynGB.                                                 *}
\textcolor{comment}{*                                                                             *}
\textcolor{comment}{* DynGB is free software: you can redistribute it and/or modify               *}
\textcolor{comment}{* it under the terms of the GNU General Public License as published by        *}
\textcolor{comment}{* the Free Software Foundation, either version 2 of the License, or           *}
\textcolor{comment}{* (at your option) any later version.                                         *}
\textcolor{comment}{*                                                                             *}
\textcolor{comment}{* Foobar is distributed in the hope that it will be useful,                   *}
\textcolor{comment}{* but WITHOUT ANY WARRANTY; without even the implied warranty of              *}
\textcolor{comment}{* MERCHANTABILITY or FITNESS FOR A PARTICULAR PURPOSE.  See the               *}
\textcolor{comment}{* GNU General Public License for more details.                                *}
\textcolor{comment}{*                                                                             *}
\textcolor{comment}{* You should have received a copy of the GNU General Public License           *}
\textcolor{comment}{* along with DynGB. If not, see <http://www.gnu.org/licenses/>.               *}
\textcolor{comment}{\(\backslash\)*****************************************************************************/}

\textcolor{preprocessor}{#include <set>}
\textcolor{preprocessor}{#include <cstdlib>}
\textcolor{preprocessor}{#include <cstring>}
\textcolor{preprocessor}{#include <iostream>}

\textcolor{keyword}{using} std::set;
\textcolor{keyword}{using} std::cout; \textcolor{keyword}{using} std::endl;

\textcolor{preprocessor}{#include "system\_constants.hpp"}

\textcolor{preprocessor}{#include "fields.hpp"}
\textcolor{preprocessor}{#include "monomial.hpp"}
\textcolor{preprocessor}{#include "polynomial.hpp"}

\textcolor{preprocessor}{#include "dynamic\_engine.hpp"}

\textcolor{preprocessor}{#include "algorithm\_buchberger\_basic.hpp"}
\textcolor{preprocessor}{#include "algorithm\_buchberger\_dynamic.hpp"}

\textcolor{keywordtype}{int} main(\textcolor{keywordtype}{int} argc, \textcolor{keywordtype}{char} *argv[]) \{
  \textcolor{keywordflow}{if} (argc != 3 or (strcmp(argv[2],\textcolor{stringliteral}{"stat"}) and strcmp(argv[2],\textcolor{stringliteral}{"dyn"}))) \{
    cout << \textcolor{stringliteral}{"need to know method (usually 2) and then if dynamic (stat or dyn)\(\backslash\)n"};
    \textcolor{keywordflow}{return} 1;
  \}
  \textcolor{comment}{// obtain method -- don't screw it up b/c we don't check it}
  \hyperlink{group___g_b_computation_ga73257b8a2d5cc826853a71b77d0cebf2}{SPolyCreationFlags} method = (\hyperlink{group___g_b_computation_ga73257b8a2d5cc826853a71b77d0cebf2}{SPolyCreationFlags} )atoi(argv[1]);
  \textcolor{keywordtype}{bool} static\_algorithm = \textcolor{keyword}{true};
  \textcolor{keywordflow}{if} (!strcmp(argv[2],\textcolor{stringliteral}{"dyn"})) static\_algorithm = \textcolor{keyword}{false};
  \textcolor{comment}{// set up the field}
  \hyperlink{group___fields_group_class_prime___field}{Prime\_Field} FF = \hyperlink{group___fields_group_class_prime___field}{Prime\_Field}(32003);
  \textcolor{keywordtype}{string} X [9] = \{ \textcolor{stringliteral}{"t"}, \textcolor{stringliteral}{"u"}, \textcolor{stringliteral}{"v"}, \textcolor{stringliteral}{"x"}, \textcolor{stringliteral}{"y"}, \textcolor{stringliteral}{"z"} \} ;
  \hyperlink{group__polygroup_class_polynomial___ring}{Polynomial\_Ring} R(6, FF, X );
  \hyperlink{group___fields_group_class_prime___field___element}{Prime\_Field\_Element} a = FF.\hyperlink{group___fields_group_ad9e3622a14de40faa75056ed40d67bb2}{unity}();
  \hyperlink{group__polygroup_class_monomial}{Monomial} one \{ 0, 0, 0, 0, 0, 0 \};
  \textcolor{comment}{// set up our polynomials}
  \textcolor{comment}{// first poly}
  \hyperlink{group__polygroup_class_monomial}{Monomial} u1 \{ 0, 1, 0, 0, 0, 0 \};
  \hyperlink{group__polygroup_class_monomial}{Monomial} v1 \{ 0, 0, 1, 0, 0, 0 \};
  \hyperlink{group__polygroup_class_monomial}{Monomial} y1 \{ 0, 0, 0, 0, 1, 0 \};
  \hyperlink{group__polygroup_class_monomial}{Monomial} M1 [] \{ u1, v1, y1, one \};
  \hyperlink{group___fields_group_class_prime___field___element}{Prime\_Field\_Element} C1 [] \{ a, a, a, -a \};
  \hyperlink{group__polygroup_class_constant___polynomial}{Constant\_Polynomial} f1(4, R, M1, C1);
  f1.\hyperlink{group__polygroup_a808018b52eca472a7a1b2995e403f35a}{sort\_by\_order}();
  \textcolor{comment}{// second poly}
  \hyperlink{group__polygroup_class_monomial}{Monomial} t1 \{ 1, 0, 0, 0, 0, 0 \};
  \hyperlink{group__polygroup_class_monomial}{Monomial} z1 \{ 0, 0, 0, 0, 0, 1 \};
  \hyperlink{group__polygroup_class_monomial}{Monomial} M2 [] \{ t1, u1, z1, one \};
  \hyperlink{group___fields_group_class_prime___field___element}{Prime\_Field\_Element} C2 [] \{ a, a*2, a, a*(-3) \};
  \hyperlink{group__polygroup_class_constant___polynomial}{Constant\_Polynomial} f2(4, R, M2, C2);
  f2.sort\_by\_order();
  \textcolor{comment}{// third poly}
  \hyperlink{group__polygroup_class_monomial}{Monomial} M3 [] \{ t1, v1, y1, one \};
  \hyperlink{group___fields_group_class_prime___field___element}{Prime\_Field\_Element} C3 [] \{ a, a*2, a, -a \};
  \hyperlink{group__polygroup_class_constant___polynomial}{Constant\_Polynomial} f3(4, R, M3, C3);
  f3.\hyperlink{group__polygroup_a808018b52eca472a7a1b2995e403f35a}{sort\_by\_order}();
  \textcolor{comment}{// fourth poly}
  \hyperlink{group__polygroup_class_monomial}{Monomial} x1 \{ 0, 0, 0, 1, 0, 0 \};
  \hyperlink{group__polygroup_class_monomial}{Monomial} M4 [] \{ t1, u1, v1, x1, y1, z1 \};
  \hyperlink{group___fields_group_class_prime___field___element}{Prime\_Field\_Element} C4 [] \{ -a, -a, -a, a, -a, -a \};
  \hyperlink{group__polygroup_class_constant___polynomial}{Constant\_Polynomial} f4(6, R, M4, C4);
  f4.\hyperlink{group__polygroup_a808018b52eca472a7a1b2995e403f35a}{sort\_by\_order}();
  \textcolor{comment}{// fifth poly}
  \hyperlink{group__polygroup_class_monomial}{Monomial} tux2 \{ 1, 1, 0, 2, 0, 0 \};
  \hyperlink{group__polygroup_class_monomial}{Monomial}  yz3 \{ 0, 0, 0, 0, 1, 3 \};
  \hyperlink{group__polygroup_class_monomial}{Monomial} M5 [] \{ tux2, yz3 \};
  \hyperlink{group___fields_group_class_prime___field___element}{Prime\_Field\_Element} C5 [] \{ a, -a*1569*((a*31250).inverse()) \};
  \hyperlink{group__polygroup_class_constant___polynomial}{Constant\_Polynomial} f5(2, R, M5, C5);
  f5.sort\_by\_order();
  \textcolor{comment}{// sixth poly}
  \hyperlink{group__polygroup_class_monomial}{Monomial} ty \{ 1, 0, 0, 0, 1, 0 \};
  \hyperlink{group__polygroup_class_monomial}{Monomial} vz \{ 0, 0, 1, 0, 0, 1 \};
  \hyperlink{group__polygroup_class_monomial}{Monomial} M6 [] \{ ty, vz \};
  \hyperlink{group___fields_group_class_prime___field___element}{Prime\_Field\_Element} C6 [] \{ -a*587*((a*15625).inverse()), a \};
  \hyperlink{group__polygroup_class_constant___polynomial}{Constant\_Polynomial} f6(2, R, M6, C6);
  f6.sort\_by\_order();
  \textcolor{comment}{// message}
  cout << \textcolor{stringliteral}{"Computing a Groebner basis for\(\backslash\)n\(\backslash\)t"} << f1
       << \textcolor{stringliteral}{",\(\backslash\)n\(\backslash\)t"} << f2
       << \textcolor{stringliteral}{",\(\backslash\)n\(\backslash\)t"} << f3
       << \textcolor{stringliteral}{",\(\backslash\)n\(\backslash\)t"} << f4
       << \textcolor{stringliteral}{",\(\backslash\)n\(\backslash\)t"} << f5
       << \textcolor{stringliteral}{",\(\backslash\)n\(\backslash\)t"} << f6
       << endl;
  \textcolor{comment}{// compute basis}
  list<Abstract\_Polynomial *> F;
  F.push\_back(&f1); F.push\_back(&f2); F.push\_back(&f3);
  F.push\_back(&f4); F.push\_back(&f5); F.push\_back(&f6);
  list<Constant\_Polynomial *> G;
  \textcolor{keywordflow}{if} (static\_algorithm) G = \hyperlink{group___g_b_computation_ga37aa7e2fec96fac6c914934a4243f603}{buchberger}(F, method, StrategyFlags::SUGAR\_STRATEGY);
  \textcolor{keywordflow}{else} G = \hyperlink{group___g_b_computation_ga40140d94eac91d7337f553d362128cb7}{buchberger\_dynamic}(
      F, method, StrategyFlags::SUGAR\_STRATEGY, \textcolor{keyword}{nullptr},
      DynamicHeuristic::ORD\_HILBERT\_THEN\_DEG
  );
  cout << \textcolor{stringliteral}{"Basis:\(\backslash\)n"};
  \textcolor{keywordflow}{for} (\hyperlink{group__polygroup_class_constant___polynomial}{Constant\_Polynomial} * g : G) \{
    cout << \textcolor{charliteral}{'\(\backslash\)t'};
    g->leading\_monomial().print(\textcolor{keyword}{true}, cout, R.\hyperlink{group__polygroup_aef9c6745956393863080422ddb8da48c}{name\_list}());
    \textcolor{keyword}{delete} g;
  \}
  cout << \textcolor{stringliteral}{"bye\(\backslash\)n"};
\}
\end{DoxyCodeInclude}
 