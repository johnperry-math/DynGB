\hypertarget{test_user_interface_8cpp-example}{}\section{test\+\_\+user\+\_\+interface.\+cpp}
This illustrates how to compute a G\"{o}bner basis of an arbitrary polynomial ideal (within the bounds this system can handle). The program is suitable for running stand-\/alone, as it prompts the user for all information, but it is probably better to pipe as input a file formatted similarly to the directory \hyperlink{}{example\+\_\+systems\+\_\+for\+\_\+user\+\_\+interface}. The basic format is\+: 
\begin{DoxyEnumerate}
\item field characteristic (should be prime) 
\item number of indeterminates 
\item whether to specify the indeterminates' names ({\ttfamily y} or {\ttfamily n}) 
\begin{DoxyEnumerate}
\item If {\ttfamily y}, follow this with the list of names 
\end{DoxyEnumerate}
\item number of generators supplied 
\item the generators, one per line, specified in expanded format (no parentheses, grouping, etc.)  
\item dynamic ({\ttfamily d}) or static ({\ttfamily s}) algorithm; if dynamic, add in this order\+: 
\begin{DoxyEnumerate}
\item which solver to use ({\ttfamily skel}, {\ttfamily ppl}, {\ttfamily glpk}) 
\item which heuristic to use ({\ttfamily h} for hilbert, {\ttfamily c} for minimal critical pairs, {\ttfamily b} for graded Betti) 
\item whether to perform a global analysis of the generators at the outset ({\ttfamily y} or {\ttfamily n}) 
\end{DoxyEnumerate}
\end{DoxyEnumerate}


\begin{DoxyCodeInclude}
\end{DoxyCodeInclude}
 