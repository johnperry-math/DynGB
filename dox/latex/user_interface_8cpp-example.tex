\hypertarget{user_interface_8cpp-example}{}\section{user\+\_\+interface.\+cpp}
This illustrates how to compute a G\"{o}bner basis of an arbitrary polynomial ideal (within the bounds this system can handle). The program is suitable for running stand-\/alone, as it prompts the user for all information, but it is probably better to pipe as input a file formatted similarly to those in the directory {\ttfamily example\+\_\+systems\+\_\+for\+\_\+user\+\_\+interface}. The basic format is\+: 
\begin{DoxyEnumerate}
\item field characteristic (should be prime) 
\item number of indeterminates 
\item whether to specify the indeterminates' names ({\ttfamily y} or {\ttfamily n}) ~\newline
--- If {\ttfamily y}, follow this with the list of names  
\item number of generators supplied 
\item the generators, one per line, specified in expanded format (no parentheses, grouping, etc.)  
\item dynamic ({\ttfamily d}) or static ({\ttfamily s}) algorithm; if dynamic, add in this order\+: 
\begin{DoxyEnumerate}
\item which solver to use ({\ttfamily skel}, {\ttfamily ppl}, {\ttfamily glpk}) 
\item which heuristic to use ({\ttfamily h} for hilbert, {\ttfamily c} for minimal critical pairs, {\ttfamily b} for graded Betti, {\ttfamily d} for minimal degree) 
\item whether to perform a global analysis of the generators at the outset ({\ttfamily y} or {\ttfamily n}) 
\end{DoxyEnumerate}
\end{DoxyEnumerate}


\begin{DoxyCodeInclude}
\textcolor{comment}{/*****************************************************************************\(\backslash\)}
\textcolor{comment}{* This file is part of DynGB.                                                 *}
\textcolor{comment}{*                                                                             *}
\textcolor{comment}{* DynGB is free software: you can redistribute it and/or modify               *}
\textcolor{comment}{* it under the terms of the GNU General Public License as published by        *}
\textcolor{comment}{* the Free Software Foundation, either version 2 of the License, or           *}
\textcolor{comment}{* (at your option) any later version.                                         *}
\textcolor{comment}{*                                                                             *}
\textcolor{comment}{* DynGB is distributed in the hope that it will be useful,                    *}
\textcolor{comment}{* but WITHOUT ANY WARRANTY; without even the implied warranty of              *}
\textcolor{comment}{* MERCHANTABILITY or FITNESS FOR A PARTICULAR PURPOSE.  See the               *}
\textcolor{comment}{* GNU General Public License for more details.                                *}
\textcolor{comment}{*                                                                             *}
\textcolor{comment}{* You should have received a copy of the GNU General Public License           *}
\textcolor{comment}{* along with DynGB. If not, see <http://www.gnu.org/licenses/>.               *}
\textcolor{comment}{\(\backslash\)*****************************************************************************/}

\textcolor{preprocessor}{#include <list>}
\textcolor{preprocessor}{#include <string>}
\textcolor{preprocessor}{#include <iostream>}

\textcolor{keyword}{using} std::list;
\textcolor{keyword}{using} std::string;
\textcolor{keyword}{using} std::cin; \textcolor{keyword}{using} std::cout; \textcolor{keyword}{using} std::endl;

\textcolor{preprocessor}{#include "system\_constants.hpp"}

\textcolor{preprocessor}{#include "monomial.hpp"}
\textcolor{preprocessor}{#include "dynamic\_engine.hpp"}
\textcolor{preprocessor}{#include "polynomial\_array.hpp"}

\textcolor{preprocessor}{#include "algorithm\_buchberger\_dynamic.hpp"}

\textcolor{keywordtype}{void} \hyperlink{group__utils_ga72d205e8226d578b892515edc527cc83}{user\_interface}() \{
  UCOEF\_TYPE p;
  cout << \textcolor{stringliteral}{"prime number for base field: "};
  cin >> p;
  NVAR\_TYPE n;
  cout << \textcolor{stringliteral}{"number of indeterminates: "};
  cin >> n;
  \textcolor{keywordtype}{char} specify\_names;
  cout << \textcolor{stringliteral}{"do you want to specify the indeterminates' names? (y/n) "};
  cin >> specify\_names;
  \textcolor{keywordtype}{string} * names = \textcolor{keyword}{new} \textcolor{keywordtype}{string}[n];
  \hyperlink{group___fields_group_class_prime___field}{Prime\_Field} F = \hyperlink{group___fields_group_class_prime___field}{Prime\_Field}(p);
  \hyperlink{group__polygroup_class_polynomial___ring}{Polynomial\_Ring} * P;
  \textcolor{keywordflow}{if} (not specify\_names) \{
    P = \textcolor{keyword}{new} \hyperlink{group__polygroup_class_polynomial___ring}{Polynomial\_Ring}(n, F);
    cout << \textcolor{stringliteral}{"indeterminate names are: "};
    \textcolor{keywordflow}{for} (NVAR\_TYPE i = 0; i < n; ++i) \{
      cout << P->\hyperlink{group__polygroup_a657caa3e9c277ca34d20cb69eed7fe05}{name}(i);
      names[i] = P->\hyperlink{group__polygroup_a657caa3e9c277ca34d20cb69eed7fe05}{name}(i);
    \}
  \} \textcolor{keywordflow}{else} \{
    \textcolor{keywordflow}{for} (NVAR\_TYPE i = 0; i < n; ++i)
      cin >> names[i];
    P = \textcolor{keyword}{new} \hyperlink{group__polygroup_class_polynomial___ring}{Polynomial\_Ring}(n, F, names);
  \}
  \textcolor{keywordtype}{unsigned} m;
  cout << \textcolor{stringliteral}{"number of polynomials: "};
  cin >> m;
  cin.ignore();
  list<Abstract\_Polynomial *> I;
  \textcolor{keywordflow}{for} (\textcolor{keywordtype}{unsigned} np = 0; np < m; ++np) \{
    cout << \textcolor{stringliteral}{"please enter polynomial #"} << np << \textcolor{stringliteral}{": (use spaces to separate factors) "};
    \textcolor{keywordtype}{string} inpoly;
    getline(cin, inpoly);
    list<Monomial> M; \textcolor{comment}{// monomials}
    list<Prime\_Field\_Element> A; \textcolor{comment}{// coefficients}
    \textcolor{keywordtype}{unsigned} nt = 0; \textcolor{comment}{// number of terms}
    \textcolor{keywordtype}{unsigned} i = 0;
    \textcolor{keywordflow}{while} (i < inpoly.size()) \{
      \textcolor{keywordtype}{bool} reading\_term = \textcolor{keyword}{true};
      COEF\_TYPE a = 1;
      EXP\_TYPE exp[n];
      \textcolor{keywordflow}{for} (NVAR\_TYPE i = 0; i < n; ++i) exp[i] = 0;
      \textcolor{keywordtype}{bool} positive = \textcolor{keyword}{true};
      \textcolor{keywordflow}{while} (inpoly[i] == \textcolor{charliteral}{' '}) ++i;
      \textcolor{keywordflow}{while} (reading\_term and i < inpoly.size()) \{
        \textcolor{keywordtype}{unsigned} j = i;
        \textcolor{keywordflow}{if} (
            inpoly[i] == \textcolor{charliteral}{'+'} or inpoly[i] == \textcolor{charliteral}{'-'} or
            (inpoly[i] >= \textcolor{charliteral}{'0'} and inpoly[i] <= \textcolor{charliteral}{'9'})
        ) \{
          \textcolor{keywordtype}{bool} positive = \textcolor{keyword}{true};
          \textcolor{keywordflow}{if} (inpoly[i] == \textcolor{charliteral}{'-'}) positive = \textcolor{keyword}{false};
          \textcolor{keywordflow}{if} (inpoly[i] == \textcolor{charliteral}{'+'} or inpoly[i] == \textcolor{charliteral}{'-'}) ++i;
          \textcolor{keywordflow}{while} (inpoly[i] == \textcolor{charliteral}{' '}) ++i;
          \textcolor{keywordflow}{if} (inpoly[i] >= \textcolor{charliteral}{'0'} and inpoly[i] <= \textcolor{charliteral}{'9'}) \{
            j = i;
            \textcolor{keywordflow}{while} (inpoly[j] >= \textcolor{charliteral}{'0'} and inpoly[j] <= \textcolor{charliteral}{'9'}) ++j;
            a *= stol(inpoly.substr(i, j - i));
            \textcolor{keywordflow}{if} (not positive) a *= -1;
            i = j;
          \}
        \}
        \textcolor{keywordflow}{else} \textcolor{keywordflow}{if} (inpoly[i] == \textcolor{charliteral}{'*'})
          ++i;
        \textcolor{keywordflow}{else} \textcolor{keywordflow}{if} (inpoly[i] == \textcolor{charliteral}{'\(\backslash\)0'})
          ++i;
        \textcolor{keywordflow}{else} \{ \textcolor{comment}{// should be characters}
          \textcolor{keywordtype}{unsigned} j = i;
          \textcolor{keywordflow}{while} ((inpoly[j] >= \textcolor{charliteral}{'a'} and inpoly[j] <= \textcolor{charliteral}{'z'}) or
                 (inpoly[j] >= \textcolor{charliteral}{'A'} and inpoly[j] <= \textcolor{charliteral}{'Z'}) or
                 (inpoly[j] >= \textcolor{charliteral}{'0'} and inpoly[j] <= \textcolor{charliteral}{'9'}))
            ++j;
          \textcolor{keywordtype}{string} var\_name = inpoly.substr(i, j - i);
          \textcolor{keywordtype}{bool} found = \textcolor{keyword}{false};
          \textcolor{keywordtype}{unsigned} k = 0;
          \textcolor{keywordflow}{for} (\textcolor{comment}{/* */}; not found and k < n; ++k)
            \textcolor{keywordflow}{if} (var\_name.compare(names[k]) == 0) \{
              found = \textcolor{keyword}{true};
              --k;
            \}
          i = j;
          \textcolor{keywordflow}{while} (inpoly[i] == \textcolor{charliteral}{' '}) ++i;
          \textcolor{keywordflow}{if} (inpoly[i] != \textcolor{charliteral}{'^'})
            exp[k] += 1;
          \textcolor{keywordflow}{else} \{
            ++i;
            \textcolor{keywordflow}{while} (inpoly[i] == \textcolor{charliteral}{' '}) ++i;
            \textcolor{keywordtype}{unsigned} j = i;
            \textcolor{keywordflow}{while} (inpoly[j] >= \textcolor{charliteral}{'0'} and inpoly[j] <= \textcolor{charliteral}{'9'}) ++j;
            exp[k] += stol(inpoly.substr(i, j - i));
            i = j;
          \}
        \}
        \textcolor{keywordflow}{while} (inpoly[i] == \textcolor{charliteral}{' '}) ++i;
        \textcolor{keywordflow}{if} (inpoly[i] == \textcolor{charliteral}{'+'} or inpoly[i] == \textcolor{charliteral}{'-'}) reading\_term = \textcolor{keyword}{false};
      \}
      ++nt;
      \textcolor{keywordflow}{if} (not positive) a *= -1;
      \hyperlink{group__polygroup_class_monomial}{Monomial} t(n, exp);
      M.push\_back(t);
      A.emplace\_back(a, &F);
    \}
    \hyperlink{group__polygroup_class_abstract___polynomial}{Abstract\_Polynomial} * p = \textcolor{keyword}{new} \hyperlink{group__polygroup_class_constant___polynomial}{Constant\_Polynomial}(*P, M, A);
    p->\hyperlink{group__polygroup_a1fcdd29c324c660ea935197c39e682f2}{sort\_by\_order}();
    cout << \textcolor{stringliteral}{"read "} << *p << endl;
    I.push\_back(p);
  \}
  \textcolor{keywordtype}{string} computation;
  \textcolor{keywordflow}{while} (
      computation.compare(\textcolor{stringliteral}{"s"}) and computation.compare(\textcolor{stringliteral}{"d"})
      and computation.compare(\textcolor{stringliteral}{"static"}) and computation.compare(\textcolor{stringliteral}{"dynamic"})
  ) \{
    cout << \textcolor{stringliteral}{"static (s) or dynamic (d) computation? "};
    getline(cin, computation);
  \}
  list<Constant\_Polynomial *> B;
  \textcolor{keywordtype}{bool} dynamic = not (computation.compare(\textcolor{stringliteral}{"d"}) and computation.compare(\textcolor{stringliteral}{"dynamic"}));
  \textcolor{keywordflow}{if} (not dynamic) \{
    B = \hyperlink{group___g_b_computation_ga37aa7e2fec96fac6c914934a4243f603}{buchberger}(I);
  \} \textcolor{keywordflow}{else} \{
    \hyperlink{group___g_b_computation_ga28fbbb9eb7d8b80ced05c8fa89b2bdac}{DynamicSolver} solver;
    \textcolor{keywordtype}{string} solver\_choice;
    \textcolor{keywordflow}{while} (
        solver\_choice.compare(\textcolor{stringliteral}{"skel"}) and solver\_choice.compare(\textcolor{stringliteral}{"glpk"})
        and solver\_choice.compare(\textcolor{stringliteral}{"ppl"}) and solver\_choice.compare(\textcolor{stringliteral}{"skeleton"})
    ) \{
      cout << \textcolor{stringliteral}{"which solver? ([skel]eton, glpk, ppl) "};
      getline(cin, solver\_choice);
    \}
    \textcolor{keywordflow}{if} (not solver\_choice.compare(\textcolor{stringliteral}{"skel"}) or not solver\_choice.compare(\textcolor{stringliteral}{"skeleton"}))
      solver = SKELETON\_SOLVER;
    \textcolor{keywordflow}{else} \textcolor{keywordflow}{if} (not solver\_choice.compare(\textcolor{stringliteral}{"glpk"}))
      solver = GLPK\_SOLVER;
    \textcolor{keywordflow}{else} \textcolor{keywordflow}{if} (not solver\_choice.compare(\textcolor{stringliteral}{"ppl"}))
      solver = PPL\_SOLVER;
    \textcolor{keywordtype}{string} heur\_choice;
    \hyperlink{group___g_b_computation_ga819b1fd40d9a40ff303df3b90647ecb0}{Dynamic\_Heuristic} heuristic;
    \textcolor{keywordflow}{while} (
      heur\_choice.compare(\textcolor{stringliteral}{"h"}) and heur\_choice.compare(\textcolor{stringliteral}{"gh"})
      and heur\_choice.compare(\textcolor{stringliteral}{"b"}) and heur\_choice.compare(\textcolor{stringliteral}{"bb"})
      and heur\_choice.compare(\textcolor{stringliteral}{"gb"}) and heur\_choice.compare(\textcolor{stringliteral}{"c"})
      and heur\_choice.compare(\textcolor{stringliteral}{"d"})
    ) \{
      cout << \textcolor{stringliteral}{"available heuristics are\(\backslash\)n"};
      cout << \textcolor{stringliteral}{"\(\backslash\)t h = hilbert with standard grading\(\backslash\)n"};
      cout << \textcolor{stringliteral}{"\(\backslash\)tgh = hilbert with order-based grading\(\backslash\)n"};
      cout << \textcolor{stringliteral}{"\(\backslash\)t b = betti with standard grading\(\backslash\)n"};
      cout << \textcolor{stringliteral}{"\(\backslash\)tbb = \(\backslash\)"big\(\backslash\)" betti with standard grading\(\backslash\)n"};
      cout << \textcolor{stringliteral}{"\(\backslash\)tgb = betti with order-based grading\(\backslash\)n"};
      cout << \textcolor{stringliteral}{"\(\backslash\)t c = minimal number of critical pairs\(\backslash\)n"};
      cout << \textcolor{stringliteral}{"\(\backslash\)t d = minimal degree, ties broken by hilbert\(\backslash\)n"};
      cout << \textcolor{stringliteral}{"which heuristic would you like? "};
      getline(cin, heur\_choice);
    \}
    \textcolor{keywordflow}{if}      (not heur\_choice.compare( \textcolor{stringliteral}{"h"})) heuristic = Dynamic\_Heuristic::ORD\_HILBERT\_THEN\_DEG;
    \textcolor{keywordflow}{else} \textcolor{keywordflow}{if} (not heur\_choice.compare(\textcolor{stringliteral}{"gh"})) heuristic = Dynamic\_Heuristic::GRAD\_HILB\_THEN\_DEG;
    \textcolor{keywordflow}{else} \textcolor{keywordflow}{if} (not heur\_choice.compare( \textcolor{stringliteral}{"b"})) heuristic = Dynamic\_Heuristic::BETTI\_HILBERT\_DEG;
    \textcolor{keywordflow}{else} \textcolor{keywordflow}{if} (not heur\_choice.compare(\textcolor{stringliteral}{"bb"})) heuristic = Dynamic\_Heuristic::BIG\_BETTI\_HILBERT\_DEG;
    \textcolor{keywordflow}{else} \textcolor{keywordflow}{if} (not heur\_choice.compare(\textcolor{stringliteral}{"gb"})) heuristic = Dynamic\_Heuristic::GRAD\_BETTI\_HILBERT\_DEG;
    \textcolor{keywordflow}{else} \textcolor{keywordflow}{if} (not heur\_choice.compare( \textcolor{stringliteral}{"c"})) heuristic = Dynamic\_Heuristic::MIN\_CRIT\_PAIRS;
    \textcolor{keywordflow}{else} \textcolor{keywordflow}{if} (not heur\_choice.compare( \textcolor{stringliteral}{"d"})) heuristic = Dynamic\_Heuristic::DEG\_THEN\_ORD\_HILBERT;
    \textcolor{keywordtype}{string} whether\_analysis;
    \textcolor{keywordtype}{bool} analyze\_first = \textcolor{keyword}{false};
    \textcolor{keywordflow}{while} (whether\_analysis.compare(\textcolor{stringliteral}{"y"}) and whether\_analysis.compare(\textcolor{stringliteral}{"n"})) \{
      cout << \textcolor{stringliteral}{"perform global analysis at the outset? (y or n) "};
      cin >> whether\_analysis;
    \}
    \textcolor{keywordflow}{if} (not whether\_analysis.compare(\textcolor{stringliteral}{"y"})) analyze\_first = \textcolor{keyword}{true};
    B = \hyperlink{group___g_b_computation_ga2c05f4e2ea8b43bb696483469f4cce83}{buchberger\_dynamic}(
        I, SPolyCreationFlags::GEOBUCKETS, StrategyFlags::SUGAR\_STRATEGY,
        \textcolor{keyword}{nullptr}, heuristic, solver, analyze\_first
    );
  \}
  cout << \textcolor{stringliteral}{"have basis with "} << B.size() << \textcolor{stringliteral}{" elements:\(\backslash\)n"};
  \textcolor{keywordflow}{for} (\textcolor{keyword}{auto} b : B) \{
    cout << b->leading\_monomial() << \textcolor{stringliteral}{", "};
    \textcolor{keyword}{delete} b;
  \}
  cout << endl;
  \textcolor{keywordflow}{for} (\textcolor{keyword}{auto} p : I) \textcolor{keyword}{delete} p;
  \textcolor{keyword}{delete} P;
  \textcolor{keyword}{delete} [] names;
\}

\textcolor{keywordtype}{int} main() \{
  \hyperlink{group__utils_ga72d205e8226d578b892515edc527cc83}{user\_interface}();
\}
\end{DoxyCodeInclude}
 