\hypertarget{test_cab_es1_8cpp-example}{}\section{test\+\_\+cab\+\_\+es1.\+cpp}
This illustrates how to compute a Gr\"{o}bner basis of the first example in \cite{CaboaraDynAlg}, \[ t^4 z b + x^3 y a ,\\ t x^8 y z + 32002 a b^4 c d e ,\\ x y^2 z^2 d + z c^2 e^2 ,\\ t x^2 y^3 z^4 + a b^2 c^3 e^2 \]


\begin{DoxyCodeInclude}
\textcolor{comment}{/*****************************************************************************\(\backslash\)}
\textcolor{comment}{* This file is part of DynGB.                                                 *}
\textcolor{comment}{*                                                                             *}
\textcolor{comment}{* DynGB is free software: you can redistribute it and/or modify               *}
\textcolor{comment}{* it under the terms of the GNU General Public License as published by        *}
\textcolor{comment}{* the Free Software Foundation, either version 2 of the License, or           *}
\textcolor{comment}{* (at your option) any later version.                                         *}
\textcolor{comment}{*                                                                             *}
\textcolor{comment}{* DynGB is distributed in the hope that it will be useful,                    *}
\textcolor{comment}{* but WITHOUT ANY WARRANTY; without even the implied warranty of              *}
\textcolor{comment}{* MERCHANTABILITY or FITNESS FOR A PARTICULAR PURPOSE.  See the               *}
\textcolor{comment}{* GNU General Public License for more details.                                *}
\textcolor{comment}{*                                                                             *}
\textcolor{comment}{* You should have received a copy of the GNU General Public License           *}
\textcolor{comment}{* along with DynGB. If not, see <http://www.gnu.org/licenses/>.               *}
\textcolor{comment}{\(\backslash\)*****************************************************************************/}

\textcolor{preprocessor}{#include <set>}
\textcolor{preprocessor}{#include <cstdlib>}
\textcolor{preprocessor}{#include <cstring>}
\textcolor{preprocessor}{#include <iostream>}

\textcolor{keyword}{using} std::set;
\textcolor{keyword}{using} std::cout; \textcolor{keyword}{using} std::endl;

\textcolor{preprocessor}{#include "system\_constants.hpp"}

\textcolor{preprocessor}{#include "fields.hpp"}
\textcolor{preprocessor}{#include "monomial.hpp"}
\textcolor{preprocessor}{#include "polynomial.hpp"}

\textcolor{preprocessor}{#include "dynamic\_engine.hpp"}

\textcolor{preprocessor}{#include "algorithm\_buchberger\_basic.hpp"}
\textcolor{preprocessor}{#include "algorithm\_buchberger\_dynamic.hpp"}

\textcolor{keywordtype}{int} main(\textcolor{keywordtype}{int} argc, \textcolor{keywordtype}{char} *argv[]) \{
  \textcolor{keywordflow}{if} (argc != 3 or (strcmp(argv[2],\textcolor{stringliteral}{"stat"}) and strcmp(argv[2],\textcolor{stringliteral}{"dyn"}))) \{
    cout << \textcolor{stringliteral}{"need to know method (usually 2) and then if dynamic (stat or dyn)\(\backslash\)n"};
    \textcolor{keywordflow}{return} 1;
  \}
  \textcolor{comment}{// obtain method -- don't screw it up b/c we don't check it}
  \hyperlink{group___g_b_computation_ga73257b8a2d5cc826853a71b77d0cebf2}{SPolyCreationFlags} method = (\hyperlink{group___g_b_computation_ga73257b8a2d5cc826853a71b77d0cebf2}{SPolyCreationFlags} )atoi(argv[1]);
  \textcolor{keywordtype}{bool} static\_algorithm = \textcolor{keyword}{true};
  \textcolor{keywordflow}{if} (!strcmp(argv[2],\textcolor{stringliteral}{"dyn"})) static\_algorithm = \textcolor{keyword}{false};
  \textcolor{comment}{// set up the field}
  \hyperlink{group___fields_group_class_prime___field}{Prime\_Field} FF(32003);
  \textcolor{keywordtype}{string} X [9] = \{\textcolor{stringliteral}{"t"}, \textcolor{stringliteral}{"x"}, \textcolor{stringliteral}{"y"}, \textcolor{stringliteral}{"z"}, \textcolor{stringliteral}{"a"}, \textcolor{stringliteral}{"b"}, \textcolor{stringliteral}{"c"}, \textcolor{stringliteral}{"d"}, \textcolor{stringliteral}{"e"}\} ;
  \hyperlink{group__polygroup_class_polynomial___ring}{Polynomial\_Ring} R(9, FF, X );
  \hyperlink{group___fields_group_class_prime___field___element}{Prime\_Field\_Element} a = FF.\hyperlink{group___fields_group_a97534ab1050f0b34023300f1bd3a97f5}{unity}();
  \textcolor{comment}{// set up our polynomials}
  \textcolor{comment}{// first poly}
  \hyperlink{group__polygroup_class_monomial}{Monomial} t11 \{ 4, 0, 0, 1, 0, 1, 0, 0, 0 \};
  \hyperlink{group__polygroup_class_monomial}{Monomial} t12 \{ 0, 3, 1, 0, 1, 0, 0, 0, 0 \};
  \hyperlink{group__polygroup_class_monomial}{Monomial} M1 [] \{ t11, t12 \};
  \hyperlink{group___fields_group_class_prime___field___element}{Prime\_Field\_Element} C1 [] \{ a, a \};
  \hyperlink{group__polygroup_class_constant___polynomial}{Constant\_Polynomial} f1(2, R, M1, C1);
  f1.\hyperlink{group__polygroup_a808018b52eca472a7a1b2995e403f35a}{sort\_by\_order}();
  \textcolor{comment}{// second poly}
  \hyperlink{group__polygroup_class_monomial}{Monomial} t21 \{ 1, 8, 1, 1, 0, 0, 0, 0, 0 \};
  \hyperlink{group__polygroup_class_monomial}{Monomial} t22 \{ 0, 0, 0, 0, 1, 4, 1, 1, 1 \};
  \hyperlink{group__polygroup_class_monomial}{Monomial} M2 [] \{ t21, t22 \};
  \hyperlink{group___fields_group_class_prime___field___element}{Prime\_Field\_Element} C2 [] \{ a, -a \};
  \hyperlink{group__polygroup_class_constant___polynomial}{Constant\_Polynomial} f2(2, R, M2, C2);
  f2.\hyperlink{group__polygroup_a808018b52eca472a7a1b2995e403f35a}{sort\_by\_order}();
  \textcolor{comment}{// third poly}
  \hyperlink{group__polygroup_class_monomial}{Monomial} t31 \{ 0, 1, 2, 2, 0, 0, 0, 1, 0 \};
  \hyperlink{group__polygroup_class_monomial}{Monomial} t32 \{ 0, 0, 0, 1, 0, 0, 2, 0, 2 \};
  \hyperlink{group__polygroup_class_monomial}{Monomial} M3 [] \{ t31, t32 \};
  \hyperlink{group___fields_group_class_prime___field___element}{Prime\_Field\_Element} C3 [] \{ a, a \};
  \hyperlink{group__polygroup_class_constant___polynomial}{Constant\_Polynomial} f3(2, R, M3, C3);
  f3.\hyperlink{group__polygroup_a808018b52eca472a7a1b2995e403f35a}{sort\_by\_order}();
  \textcolor{comment}{// fourth poly}
  \hyperlink{group__polygroup_class_monomial}{Monomial} t41 \{ 1, 2, 3, 4, 0, 0, 0, 0, 0 \};
  \hyperlink{group__polygroup_class_monomial}{Monomial} t42 \{ 0, 0, 0, 0, 1, 2, 3, 0, 2 \};
  \hyperlink{group__polygroup_class_monomial}{Monomial} M4 [] \{ t41, t42 \};
  \hyperlink{group___fields_group_class_prime___field___element}{Prime\_Field\_Element} C4 [] \{ a, a \};
  \hyperlink{group__polygroup_class_constant___polynomial}{Constant\_Polynomial} f4(2, R, M4, C4);
  f4.\hyperlink{group__polygroup_a808018b52eca472a7a1b2995e403f35a}{sort\_by\_order}();
  \textcolor{comment}{// message}
  cout << \textcolor{stringliteral}{"Computing a Groebner basis for\(\backslash\)n\(\backslash\)t"} << f1
       << \textcolor{stringliteral}{",\(\backslash\)n\(\backslash\)t"} << f2
       << \textcolor{stringliteral}{",\(\backslash\)n\(\backslash\)t"} << f3
       << \textcolor{stringliteral}{",\(\backslash\)n\(\backslash\)t"} << f4
       << endl;
  \textcolor{comment}{// compute basis}
  list<Abstract\_Polynomial *> F;
  F.push\_back(&f1); F.push\_back(&f2); F.push\_back(&f3); F.push\_back(&f4);
  list<Constant\_Polynomial *> G;
  \textcolor{keywordflow}{if} (static\_algorithm) G = \hyperlink{group___g_b_computation_ga37aa7e2fec96fac6c914934a4243f603}{buchberger}(F, method, StrategyFlags::SUGAR\_STRATEGY);
  \textcolor{keywordflow}{else} G = \hyperlink{group___g_b_computation_ga2c05f4e2ea8b43bb696483469f4cce83}{buchberger\_dynamic}(
        F, method, StrategyFlags::SUGAR\_STRATEGY, \textcolor{keyword}{nullptr},
        Dynamic\_Heuristic::ORD\_HILBERT\_THEN\_DEG
  );
  cout << \textcolor{stringliteral}{"Basis:\(\backslash\)n"};
  \textcolor{keywordflow}{for} (\hyperlink{group__polygroup_class_constant___polynomial}{Constant\_Polynomial} * g : G) \{
    cout << \textcolor{charliteral}{'\(\backslash\)t'};
    g->leading\_monomial().print(\textcolor{keyword}{true}, cout, R.\hyperlink{group__polygroup_aef9c6745956393863080422ddb8da48c}{name\_list}());
  \}
  cout << \textcolor{stringliteral}{"bye\(\backslash\)n"};
\}
\end{DoxyCodeInclude}
 