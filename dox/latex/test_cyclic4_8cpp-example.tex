\hypertarget{test_cyclic4_8cpp-example}{}\section{test\+\_\+cyclic4.\+cpp}
This illustrates how to compute a Gr\"{o}bner basis of the Cyclic-\/4 system \[ x_1 + x_2 + x_3 + x_4,\\ x_1 x_2 + x_2 x_3 + x_3 x_4 + x_4 x_1,\\ x_1 x_2 x_3 + x_2 x_3 x_4 + x_3 x_4 x_1 + x_4 x_1 x_2,\\ x_1 x_2 x_3 x_4 - 1 \] using this package.


\begin{DoxyCodeInclude}
\textcolor{preprocessor}{#include <set>}
\textcolor{preprocessor}{#include <iostream>}

\textcolor{comment}{/*****************************************************************************\(\backslash\)}
\textcolor{comment}{* This file is part of DynGB.                                                 *}
\textcolor{comment}{*                                                                             *}
\textcolor{comment}{* DynGB is free software: you can redistribute it and/or modify               *}
\textcolor{comment}{* it under the terms of the GNU General Public License as published by        *}
\textcolor{comment}{* the Free Software Foundation, either version 2 of the License, or           *}
\textcolor{comment}{* (at your option) any later version.                                         *}
\textcolor{comment}{*                                                                             *}
\textcolor{comment}{* DynGB is distributed in the hope that it will be useful,                    *}
\textcolor{comment}{* but WITHOUT ANY WARRANTY; without even the implied warranty of              *}
\textcolor{comment}{* MERCHANTABILITY or FITNESS FOR A PARTICULAR PURPOSE.  See the               *}
\textcolor{comment}{* GNU General Public License for more details.                                *}
\textcolor{comment}{*                                                                             *}
\textcolor{comment}{* You should have received a copy of the GNU General Public License           *}
\textcolor{comment}{* along with DynGB. If not, see <http://www.gnu.org/licenses/>.               *}
\textcolor{comment}{\(\backslash\)*****************************************************************************/}

\textcolor{keyword}{using} std::set;
\textcolor{keyword}{using} std::cout; \textcolor{keyword}{using} std::endl;

\textcolor{preprocessor}{#include "algorithm\_buchberger\_basic.hpp"}
\textcolor{preprocessor}{#include "fields.hpp"}
\textcolor{preprocessor}{#include "monomial.hpp"}
\textcolor{preprocessor}{#include "polynomial.hpp"}

\textcolor{keywordtype}{int} main(\textcolor{keywordtype}{int} argc, \textcolor{keywordtype}{char} *argv[]) \{
  \textcolor{comment}{// set up the field}
  \hyperlink{group___fields_group_class_prime___field}{Prime\_Field} F43 = \hyperlink{group___fields_group_class_prime___field}{Prime\_Field}(43);
  \hyperlink{group__polygroup_class_polynomial___ring}{Polynomial\_Ring} R(4, F43);
  \hyperlink{group___fields_group_class_prime___field___element}{Prime\_Field\_Element} a = F43.\hyperlink{group___fields_group_a97534ab1050f0b34023300f1bd3a97f5}{unity}();
  \textcolor{comment}{// set up our polynomials}
  \textcolor{comment}{// first poly: linear}
  \textcolor{keywordtype}{unsigned} e1 [] \{1,0,0,0\}; \hyperlink{group__polygroup_class_monomial}{Monomial} t1(4, e1);
  \textcolor{keywordtype}{unsigned} e2 [] \{0,1,0,0\}; \hyperlink{group__polygroup_class_monomial}{Monomial} t2(4, e2);
  \textcolor{keywordtype}{unsigned} e3 [] \{0,0,1,0\}; \hyperlink{group__polygroup_class_monomial}{Monomial} t3(4, e3);
  \textcolor{keywordtype}{unsigned} e4 [] \{0,0,0,1\}; \hyperlink{group__polygroup_class_monomial}{Monomial} t4(4, e4);
  \hyperlink{group__polygroup_class_monomial}{Monomial} M1 [] \{ t1, t2, t3, t4 \};
  \hyperlink{group___fields_group_class_prime___field___element}{Prime\_Field\_Element} C1 [] \{ a, a, a, a \};
  \hyperlink{group__polygroup_class_constant___polynomial}{Constant\_Polynomial} f1(4, R, M1, C1);
  \textcolor{comment}{// second poly: quadratic}
  \textcolor{keywordtype}{unsigned} e5 [] \{1,1,0,0\}; \hyperlink{group__polygroup_class_monomial}{Monomial} t5(4, e5);
  \textcolor{keywordtype}{unsigned} e6 [] \{0,1,1,0\}; \hyperlink{group__polygroup_class_monomial}{Monomial} t6(4, e6);
  \textcolor{keywordtype}{unsigned} e7 [] \{0,0,1,1\}; \hyperlink{group__polygroup_class_monomial}{Monomial} t7(4, e7);
  \textcolor{keywordtype}{unsigned} e8 [] \{1,0,0,1\}; \hyperlink{group__polygroup_class_monomial}{Monomial} t8(4, e8);
  \hyperlink{group__polygroup_class_monomial}{Monomial} M2 [] \{ t5, t6, t7, t8 \};
  \hyperlink{group___fields_group_class_prime___field___element}{Prime\_Field\_Element} C2 [] \{ a, a, a, a \};
  \hyperlink{group__polygroup_class_constant___polynomial}{Constant\_Polynomial} f2(4, R, M2, C2);
  f2.\hyperlink{group__polygroup_a808018b52eca472a7a1b2995e403f35a}{sort\_by\_order}();
  \textcolor{comment}{// third poly: cubic}
  \textcolor{keywordtype}{unsigned} e9  [] \{1,1,1,0\}; \hyperlink{group__polygroup_class_monomial}{Monomial} t9 (4, e9);
  \textcolor{keywordtype}{unsigned} e10 [] \{0,1,1,1\}; \hyperlink{group__polygroup_class_monomial}{Monomial} t10(4, e10);
  \textcolor{keywordtype}{unsigned} e11 [] \{1,0,1,1\}; \hyperlink{group__polygroup_class_monomial}{Monomial} t11(4, e11);
  \textcolor{keywordtype}{unsigned} e12 [] \{1,1,0,1\}; \hyperlink{group__polygroup_class_monomial}{Monomial} t12(4, e12);
  \hyperlink{group__polygroup_class_monomial}{Monomial} M3 [] \{ t9, t10, t11, t12 \};
  \hyperlink{group___fields_group_class_prime___field___element}{Prime\_Field\_Element} C3 [] \{ a, a, a, a \};
  \hyperlink{group__polygroup_class_constant___polynomial}{Constant\_Polynomial} f3(4, R, M3, C3);
  f3.\hyperlink{group__polygroup_a808018b52eca472a7a1b2995e403f35a}{sort\_by\_order}();
  \textcolor{comment}{// fourth poly: quartic}
  \textcolor{keywordtype}{unsigned} e13 [] \{1,1,1,1\}; \hyperlink{group__polygroup_class_monomial}{Monomial} t13(4, e13);
  \hyperlink{group__polygroup_class_monomial}{Monomial} t14(4);
  \hyperlink{group__polygroup_class_monomial}{Monomial} M4 [] \{ t13, t14 \};
  \hyperlink{group___fields_group_class_prime___field___element}{Prime\_Field\_Element} C4 [] \{ a, -a \};
  \hyperlink{group__polygroup_class_constant___polynomial}{Constant\_Polynomial} f4(2, R, M4, C4);
  f4.\hyperlink{group__polygroup_a808018b52eca472a7a1b2995e403f35a}{sort\_by\_order}();
  \textcolor{comment}{// message}
  cout << \textcolor{stringliteral}{"Computing a Groebner basis for "} << f1 << \textcolor{stringliteral}{", "} << f2
       << \textcolor{stringliteral}{", "} << f3 << \textcolor{stringliteral}{", "} << f4 << endl;
  \textcolor{comment}{// compute basis}
  set<Abstract\_Polynomial *> F;
  F.insert(&f1); F.insert(&f2); F.insert(&f3); F.insert(&f4);
  list<Constant\_Polynomial *> G = \hyperlink{group___g_b_computation_ga37aa7e2fec96fac6c914934a4243f603}{buchberger}(F);
  cout << \textcolor{stringliteral}{"Basis:\(\backslash\)n"};
  \textcolor{keywordflow}{for} (list<Constant\_Polynomial *>::iterator g = G.begin(); g != G.end(); ++g)
    cout << \textcolor{charliteral}{'\(\backslash\)t'} << *(*g) << endl;
  cout << \textcolor{stringliteral}{"bye\(\backslash\)n"};
\}
\end{DoxyCodeInclude}
 